
\chapter{Introduction}
\label{chap:intro}

\section{Welcome}
\label{sec:welcome}

We human beings find it very difficult to completely clear our minds. That means you have been thinking of something nearly every waking moment since you began to think. If we assume eight hours of sleep every night, then that comes to just over 7,000,000 minutes of thinking in 20 years. Surely, after that much time spent doing something, you ought to have become pretty good at it. So, why should you consider reading a book or taking a course that claims to teach you how to do something you've been doing for years?

Well, as the old saying goes, I have some good news and some bad news. The bad news is that we're just not very good at thinking carefully. Some things are easy enough for us -- you probably don't have a problem when it comes to deciding whether you should step out in front of a truck. On the other hand, when it comes to difficult, tricky subjects, we're often more likely to come up with wrong answers as right ones.

For example, consider this problem:

\begin{quote}
A ball and bat together cost \$1.10, and the bat costs \$1.00 more than the ball does. How much does the ball cost?
\end{quote}

I'll give you some time to think about your answer, although you shouldn't need much time. So, whenever you're ready, turn the page\ldots

\newpage

Was your answer ten cents? That's the most common answer, but it's also clearly wrong. If the ball costs ten cents, and the bat costs a dollar more than the ball, then the bat costs \$1.10 and the total would be \$1.20. The right answer has to be five cents: \$1.05 + \$.05 = \$1.10. Even though it's not a difficult problem, most people get it wrong.

On the other hand, maybe we get it wrong \emph{because} it's not a difficult
problem. It looks so simple that we answer it without thinking about it.
When we don't reason carefully about a problem, our minds provide us
with an automatic answer. In some situations, the automatic answer
provided by the mind is very likely to be true. In others, it is very
likely to be false.

At this point, you might be asking yourself, \enquote{So what?} What's the worst
that could happen, maybe getting a nickel extra in change when you buy
the ball? This still doesn't justify taking a whole course to learn how
to think better, does it?

Consider one more example:

\begin{quote}


1\% of women at age forty who participate in routine screening have
breast cancer. 80\% of women with breast cancer will get positive
mammographies. 9.6\% of women without breast cancer will also get
positive mammographies. A woman in this age group had a positive
mammography in a routine screening. What is the probability that she
actually has breast cancer?
\end{quote}

Unlike the ball and the bat, being wrong in this case could have drastic
consequences---if the doctor guessed too low, then the patient likely
did not receive the treatment she needed. If the doctor guessed too
high, then the patient may received radical treatments that she didn't
need, unnecessary radiation, chemotherapy, or even a radical mastectomy.

Doctors have to make these diagnoses all the time, so surely they would
be good at correctly estimating the patient's likelihood of having the
disease. Most doctors estimated that, in this problem, the patient's
chances of having breast cancer are somewhere between 70 and 80\%. Only
15\% of doctors surveyed were correct, however. Surprisingly, the right
answer is 7.8\%, a mere one-tenth of the estimates by the medical
professionals.\footnote{That doesn't mean that the test should be ignored. It just means
    that the doctors should not immediately begin dangerous treatments.
    What is warranted is further testing to lessen the chances of a
    false positive.}

So, how does one avoid making such mistakes? The best way is to become a
better critical thinker. You've taken the right initial steps by reading
this book and taking this class.

\section{What is Critical Thinking?}
\label{sec:what-is-ct}



There are probably as many definitions of critical thinking as there
books and articles on the subject. Here is a quick working definition
that will suit our purposes:

\begin{description}
\item[Critical Thinking] Thinking clearly and carefully about what to believe or do in a way that is likely to produce a true belief or right action, if there is one. 
\end{description}





There are a few things to note about this definition. First, critical
thinking is practical. It is designed to produce a particular outcome,
either a belief or an action. The goal is to gain true beliefs while
avoiding false beliefs, or to do right actions and avoid doing wrong
ones. At this point, we don't need to rehash old disputes about the
nature of truth or morality, our ordinary understanding of the two
concepts will be fine.

Second, there is nothing that we can do that guarantees a true
belief---at best, we only get likelihood. Nevertheless, when we use the
tools of critical thinking, we will be more likely to get to the truth
than had we not used those tools.

Third, it is important to note that not every question has an answer
that we can know just by thinking carefully about the problem. There are
some questions that have right answers, but just cannot be known by us.
How many life-supporting planets are there in the universe? We know
there is at least one, but we don't yet have the ability to if there are
any others. There are other questions for which there is no right
answer, at least not in the objective sense. What kind of ice cream is
best? You may have your preference, and I might have a different view.
Is either of us wrong? Don't hold up the line in the ice cream shop
telling yourself, \enquote{I know I like chocolate better than vanilla, but
which one is \emph{really} the best?} There is no best in this case, so order
whatever you want.

This is a classic case of what philosophers call purely subjective. A
subjective truth is one that is dependent on what a person prefers,
thinks, believes, etc. Objective truths are true independently of what
anyone thinks, believes, perceives, etc. Critical thinking won't help us
answer the subjective questions, but we don't really have problems with
those. In those cases, it's good enough just to report how we feel,
since that is what makes those subjective beliefs true. Critical
thinking, however, will help us decide if a question is objective or
subjective, and if objective, if it can be answered.

\section{The Tools of Critical Thinking}
\label{sec:tools-ct}

It's not enough to tell someone to think clearly and carefully---we have
to know what clear and careful thinking, and clear and careful thinkers,
look like. Critical thinking is a skill, and like many other skills, it
involves the skilled use of tools. One set of tools will be no surprise;
they are the tools of logic. Good critical thinkers can

\begin{enumerate}

\item   Identify arguments in propositional and categorical logic
\item   Evaluate arguments using truth tables and Venn diagrams
\item   Use the basic rules of probability, and
\item   Identify common logical fallacies.
\end{enumerate}

Another set of tools has been provided by cognitive psychologists.
Critical thinkers need to understand how the human mind works,
especially the systematic ways that the mind is misled. So, critical
thinkers must

\begin{enumerate}
\item   Understand common cognitive biases,
\item   Be aware of the ways that people try to mislead us,
\item   Know the situations in which we tend to reason badly.
\end{enumerate}

Finally, I think that it's not enough that critical thinkers understand
logic. It's not even enough to understand logic and cognitive
psychology. I think there is a moral, or value component to critical
thinking as well. To become a critical thinker is to become a certain  kind of person, a person of  intellectual virtue. So, we will
discuss the importance and roles of such virtues as

\begin{enumerate}
\item   Open mindedness
\item   Intellectual courage
\item   Intellectual humility
\item   Attentiveness
\item   Fairness
\item   Perseverance
\item   Firmness
\end{enumerate}

So, by the end of this book, and by the end of your course, I hope that
you are well on the road to acquiring these skills. Like any other
skills, they cannot be acquired without practice. You will not become a
perfect critical thinker in a semester, maybe not even over the course
of a lifetime. You can, however, take some significant steps on a
journey that leads to one of the most important destinations ever: the
truth.


%%% Local Variables:
%%% mode: latex
%%% TeX-master: "ct-text"
%%% End:
